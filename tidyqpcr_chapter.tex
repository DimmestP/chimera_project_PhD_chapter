%
% Sample SBC book chapter
%
% This is a public-domain file.
%
% Charset: ISO8859-1 (latin-1) áéíóúç
%
\documentclass{SBCbookchapter}
\usepackage[utf8]{inputenc}
\usepackage[T1]{fontenc}
\usepackage[brazil,english]{babel}
\usepackage{graphicx}
\usepackage{array}

\setcounter{chapter}{4}

\author{Sam Haynes}
\title{{Limitations} of composability of cis-regulatory elements in messenger RNA}

\begin{document}
\maketitle

\begin{abstract}
This chapter is about finding short cis-regulatory motifs in the 3'UTR of mRNA transcripts and detecting if their regulatory behaviour is dependent on the context of the whole transcript. This could be the context of the 5'UTR, ORF or the 3'UTR. We also introduced two motifs at the same time to see if they interact with each other. To begin, a literature search found 69 motifs previously identified as being enriched in transcripts that bind proteins or are highly expressed. Then, a linear model predicting transcript half life using codon usage, 3'UTR length and the presence or absence of these motifs was trained on two independent half life data sets. Using a greedy algorithm that maximised the AIC of the model, the 69 motifs were shortlisted to just 4 likely contributors. These 4 chosen motifs were inserted into two different native terminators and removed from one terminator that they appear in natively. Each motif-terminator set was also pair with three different promoters creating a library of 20 constructs. Using qPCR to detect changes in the expression of these constructs we then determined whether these motifs have the same effect in different contexts or whether they contribute the same across transcripts. We confirmed these effects are repeated in protein expression and in RNA-Seq expression. Finally we attempted to see if these transcripts had different PolyA sites in the total and decay populations but found different in only one case.
\end{abstract}

\section{Research Software}

\subsection{Software development practices}

\subsubsection{Computation and the environment}

\subsection{GUIs or CLIs}

\begin{itemize}
    \item Since the macintosh there has been a continued shift between GUI and CLI statistical software taking the lead. 
    \item GUIs tend to be associated a dramatically lower learning curve but larger development and maintenance costs.
    \item Better user experience in a quality GUI than CLI, however do GUI developers prioritise ease of use over quality of analysis by ignoring key steps (such as quality control).
    \item GUIs are also limited by the number of ways a user can interact with the screen (interactive operations), drop down boxes quickly become unwieldy if they list every possible action to take on an object. 
    \item CLIs are often quicker to develop (so can include the latest statistical methods), easier to scale to larger work loads, better reproducibility and cheaper to maintain.
    \item Without a unified framework to conduct statistical analysis (how you define models and manipulate data) there cannot be an intuitive structure to create and expand a GUI.

\end{itemize}

\section{Software documentation}

\subsection{Why is research software documentation important}

\bibliographystyle{apalike}
\bibliography{sbc-template}

\end{document}